%-------------------------------------------------------------------------------
%	SECTION TITLE
%-------------------------------------------------------------------------------
\cvsection{Work Experience}


%-------------------------------------------------------------------------------
%	CONTENT
%-------------------------------------------------------------------------------
\begin{cventries}

%---------------------------------------------------------
% \cventrytwopositions
%     {Data Scientist II - Supply Chain Forecasting}
%     {Chewy}
%     {Bellevue, WA}
%     {Jan 2022 - Dec 2023}
%     {
%           \begin{cvitems}    
%         \item {Architected and implemented an end-to-end ML pipeline for Chewy Autoship orders. Combining PySpark for ETL processes via AWS Glue, model training and inference via AWS Sagemaker, and orchestration through Apache Airflow dags, the project outperforms the existing heuristic-based model by 3-5\% MAPE, but also works as a scalable framework for future e2e ML tasks.}    
%         \item {Engineered and deployed a Hierarchical Reconciliation framework as a callable API, allowing for seamless integration with existing systems for reconciling forecasts across multiple granularities. This framework is now used by a majority of our production ML pipelines.}
%         % \item {Enhanced the core SKU demand model by integrating promotion features, fine-tuning model accuracy and responsiveness to market trends and promotional impacts.}
%         \item {Redesigned unconstrained demand model, which is central to our core SKU demand model. The updated framework includes an ensemble of clickstream heuristic models and self-attention-based transformer models allowing for more accurate and insightful demand forecasting and reducing out-of-stock MAPE by up to 10\%.}
%         % \item {Developed a Hierarchical Reconciliation framework to ensure coherency across forecasting granularities, based on latest research and rigorous unit testing.}
%         % \item {Led effort into topic demand forecasting, employing a temporal knowledge graph approach to capture seasonal and thematic trends effectively.}
%     \end{cvitems}
%     }
%     {Machine Learning Engineer III - Supply Chain Forecasting}
%     {Dec 2023 - Present}
%     {
%       \begin{cvitems} % Description(s) of tasks/responsibilities
%         % \item {Revamped Load Forecasting systems for over 30 clients reducing MAPE by 4-10\%, implementing an ensembling approach using models like RNN/CNN and XGBoost with k-fold cross-validation and outlier detection.}
%         \item {In my expanded role, I've spearheaded the enhancement of many of our production pipelines, improving our team's flagship products through meticulous testing and scalability improvements.}
%         \item {Led effort into topic demand forecasting, i.e. topics like "winter" or "nutrition", employing a temporal knowledge graph approach to capture both seasonal and thematic trends effectively.}
%         \item {Led effort into customer behavior analytics, leverage big data technologies like Spark and Hadoop to draw conclusions from our massive user clickstream dataset.}
%       \end{cvitems}
%     }
  \cventry
    {Data Scientist II - Supply Chain Forecasting} % Job title
    {Chewy} % Organization
    {Bellevue, WA} % Location
    {Jan 2022 - Present} % Date(s)
    {
      \begin{cvitems}    
        \item {Architected and deployed an end-to-end ML pipeline for forecasting Chewy Autoship orders. Combining PySpark for ETL processes via AWS Glue, model training and inference via AWS Sagemaker, and orchestration through Apache Airflow dags, the pipeline outperformed existing heuristic-based model by 3-5\% MAPE.}    
        \item {Architected and deployed a Hierarchical Reconciliation framework, allowing for seamless integration with existing forecasting systems and providing a robust and scalable solution for reconciling forecasts across multiple granularities. This framework is now used by a majority of our production ML pipelines.}
        \item {Enhanced the core SKU demand model by integrating promotion features, fine-tuning model accuracy and responsiveness to market trends and promotional impacts.}
        \item {Redesigned unconstrained demand model, which is central to our core SKU demand model. The updated framework includes an ensemble of conversation-rate heuristic models and tree-based models, reducing out-of-stock MAPE by up to 10\%.}
        \item {Led effort into topic demand forecasting, employing a temporal knowledge graph approach to capture seasonal and thematic trends effectively.}
    \end{cvitems}
    }

    \cventry
    {Data Scientist - Analytics} % Job title
    {The Energy Authority} % Organization
    {Bellevue, WA} % Location
    {Sep 2019 - Dec 2022} % Date(s)
    {
      \begin{cvitems} % Description(s) of tasks/responsibilities
        % \item {Revamped Load Forecasting systems for over 30 clients reducing MAPE by 4-10\%, implementing an ensembling approach using models like RNN/CNN and XGBoost with k-fold cross-validation and outlier detection.}
        \item {Overhauled Hydro Forecasting pipeline by integrating new API data feeds from NW river system and plant-level generation, achieving a 5\% reduction in system-level MAPE.}
        \item {Lead transition to new CI/CD system for running containerized applications, enabling streamlined deployments and builds and establishing new team standards for project version control and production deployment.}
        \item {Worked with Software Engineering team to develop model failsafe system providing forecasts to real-time traders during critical data feed failures.}
      \end{cvitems}
    }

  % \cventrytwopositions
  %   {Analytics Intern}
  %   {The Energy Authority}
  %   {Bellevue, WA}
  %   {Summer 2019}
  %   {
  %     \begin{cvitems} % Description(s) of tasks/responsibilities
  %       \item {Supported other analysts in the maintenance and development of production models and error diagnostics.}
  %       \item {Redesigned several existing web applications and frameworks allowing for faster data retrieval and an integrated system for generating reports for client presentations including changes to the long-term forecast resulting in an average of 10-20 hours saved in annual work-hours per client.}
  %     \end{cvitems}
  %   }
  %   {Data Scientist - Analytics}
  %   {Sep 2019 - Dec 2022}
  %   {
  %     \begin{cvitems} % Description(s) of tasks/responsibilities
  %       % \item {Revamped Load Forecasting systems for over 30 clients reducing MAPE by 4-10\%, implementing an ensembling approach using models like RNN/CNN and XGBoost with k-fold cross-validation and outlier detection.}
  %       \item {Overhauled Hydro Forecasting pipeline by integrating new API data feeds from NW river system and plant-level generation, achieving a 5\% reduction in system-level MAPE.}
  %       \item {Lead transition to new CI/CD system for running containerized applications, enabling streamlined deployments and builds and establishing new team standards for project version control and production deployment.}
  %       \item {Worked with Software Engineering team to develop model failsafe system providing forecasts to real-time traders during critical data feed failures.}
  %     \end{cvitems}
  %   }

%---------------------------------------------------------
  % \cventry
  %   {Business Systems Analyst Intern} % Job title
  %   {Equinix} % Organization
  %   {Sunnyvale, CA} % Location
  %   {Summer 2018} % Date(s)
  %   {
  %     \begin{cvitems} % Description(s) of tasks/responsibilities
  %       \item {Created and updated IT application runbooks for core applications.}
  %       \item {Built reports and dashboards to measure success and operational efficiency of organizational initiatives.}
  %     \end{cvitems}
  %   }

%---------------------------------------------------------
\end{cventries}
